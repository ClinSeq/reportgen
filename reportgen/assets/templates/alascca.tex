\documentclass[10pt]{article}
\usepackage[table]{xcolor}
\usepackage{booktabs}
\usepackage[top=0.75cm,lmargin=1cm,rmargin=1cm,bottom=1cm]{geometry}
\usepackage[utf8]{inputenc}
\usepackage{graphicx}
\usepackage{longtable}
\usepackage{changepage}
\usepackage{fancyhdr}
\usepackage[many]{tcolorbox}
\usepackage{colortbl}
\usepackage{multicol}
\usepackage{multirow}
\usepackage{setspace}

\setlength{\parindent}{0pt}


\definecolor{grey}{rgb}{0.2, 0.2, 0.2}

\definecolor{lightgrey}{rgb}{0.7, 0.7, 0.7}

\definecolor{ratherlightgrey}{rgb}{0.9, 0.9, 0.9}

\renewcommand*{\familydefault}{\sfdefault}
\renewcommand{\sfdefault}{phv}

\begin{document}

%# Nesting the entire document body in a for loop, in order to generate
%# a separate report for each return address:
\BLOCK{ for address in metaJSON.return_addresses }

\pagenumbering{gobble}

\vspace{0cm}

\begin{center}
\huge{ClinSeq Analysrapport}
\end{center}

\vspace{-0.3cm}

\begin{tabular}[t]{  c  c  }
  \BLOCK{ for logo in docFormat.logo_files }
    \includegraphics[width=35mm]{\VAR{logo}}
    \BLOCK{ if not loop.last } &
    \BLOCK{ endif }
  \BLOCK{ endfor }
\end{tabular}

\vspace{0.3cm}

\rowcolors{1}{}{}
\begin{tabular}{ l l }
\multirow{2}{10.5cm}{\begin{tabular}{l}Personnummer \VAR{metaJSON.personnummer} \\
Analys genomförd \VAR{reportDate}\\
\end{tabular}} &
\multirow{4}{7cm}{
\begin{tabular}{l}\VAR{address.attn}\\
\VAR{address.line1}\\
\VAR{address.line2}\\
\VAR{address.line3}\\
\end{tabular}} \\
 & \\
 & \\
 & \\
\end{tabular}

\vspace{1cm}

\onehalfspacing
{
Blodprov taget \VAR{metaJSON.blood_sample_date}, remiss-ID \VAR{metaJSON.blood_referral_ID}, etikett \VAR{metaJSON.blood_sample_ID} \\
Tumörprov taget \VAR{metaJSON.tumor_sample_date}, remiss-ID \VAR{metaJSON.tumor_referral_ID}, etikett \VAR{metaJSON.tumor_sample_ID} \\
}
\par
\singlespacing

\setlength{\fboxsep}{10pt}
\setlength{\fboxrule}{3pt}

\rowcolors{1}{ratherlightgrey}{}
\fcolorbox{blue}{white}{%
  \parbox{18.9cm}{%
\centering
{\Large Randomisering till ALASCCA-studien} \par
\BLOCK{ set class_a_box = docFormat.unchecked }
\BLOCK{ set class_b_box = docFormat.unchecked }
\BLOCK{ set alascca_no_mutations_box = docFormat.unchecked }
\BLOCK{ set alascca_not_determined_box = docFormat.unchecked }
\BLOCK{ if genomicJSON.alascca_class_report.alascca_class == "Mutation class A" }
  \BLOCK{ set class_a_box = docFormat.checked }
\BLOCK{ elif genomicJSON.alascca_class_report.alascca_class == "Mutation class B" }
  \BLOCK{ set class_b_box = docFormat.checked }
\BLOCK{ elif genomicJSON.alascca_class_report.alascca_class == "Not mutated" }
  \BLOCK{ set alascca_no_mutations_box = docFormat.checked }
\BLOCK{ elif genomicJSON.alascca_class_report.alascca_class == "Not determined" }
  \BLOCK{ set alascca_not_determined_box = docFormat.checked }
\BLOCK{ endif }

\vspace{0.2cm}
\tcbox[left=0mm,right=-1mm,top=0mm,bottom=0mm,boxsep=0mm,
       boxrule=0.4pt, colframe=grey, colback=white]
{
\begin{tabular}{l l}
\includegraphics{\VAR{class_a_box}} & Mutationsklass A, patienten kan randomiseras \\
\includegraphics{\VAR{class_b_box}} & Mutationsklass B, patienten kan randomiseras \\
& \\
\includegraphics{\VAR{alascca_no_mutations_box}} & Inga mutationer, patienten kan \emph{ej} randomiseras \\
\includegraphics{\VAR{alascca_not_determined_box}} & Ej utförd/ej bedömbar, patienten kan \emph{ej} randomiseras \\
\end{tabular}
}
}
}
\begin{center}
{\Large Övrig information från ClinSeq-profil}\label{clinseq-report}
\end{center}

\begin{minipage}{.3\linewidth}
\fcolorbox{lightgrey}{white}{%
  \parbox[t][3cm][t]{4.1cm}{%
\centering

{{\large Mikrosatellitinstabilitet\\(MSI)}} \par
\vspace{0.4cm}

\BLOCK{ set mss_box = docFormat.unchecked }
\BLOCK{ set msi_box = docFormat.unchecked }
\BLOCK{ set msi_not_determined_box = docFormat.unchecked }
\BLOCK{ if genomicJSON.msi_report.msi_status == "MSS/MSI-L" }
  \BLOCK{ set mss_box = docFormat.checked }
\BLOCK{ elif genomicJSON.msi_report.msi_status == "MSI-H" }
  \BLOCK{ set msi_box = docFormat.checked }
\BLOCK{ elif genomicJSON.msi_report.msi_status == "Not determined" }
  \BLOCK{ set msi_not_determined_box = docFormat.checked }
\BLOCK{ endif }

\tcbox[left=0mm,right=-1mm,top=0mm,bottom=0mm,boxsep=0mm,
  boxrule=0.4pt, colframe=grey, colback=white]% set to your wish
{
  \begin{tabular}{l | l}
MSI-H\textsuperscript{1} & \includegraphics{\VAR{msi_box}} \\
MSS/MSI-L\textsuperscript{2} & \includegraphics{\VAR{mss_box}} \\
Ej bedömbar & \includegraphics{\VAR{msi_not_determined_box}} \\
  \end{tabular}
}
}}
\end{minipage}
\begin{minipage}{.7\linewidth}
\fcolorbox{lightgrey}{white}{%
  \parbox[t][3cm][t]{13.1cm}{%
    \centering

    \rowcolors{2}{}{ratherlightgrey}

    {{\large Övriga mutationer}} \par
\vspace{0.3cm}

\tcbox[left=0mm,right=0mm,top=0mm,bottom=0mm,boxsep=0mm,
  boxrule=0.4pt, colframe=grey, colback=white]% set to your wish
{

  \begin{tabular}{l | l | l | l | l}
  Gen & Mutation & Ej mutation & Ej bedömbar & Kommentar \tabularnewline
\arrayrulecolor{grey}\hline
\arrayrulecolor{grey}\hline

  \BLOCK{ for gene in genomicJSON.simple_somatic_mutations_report.keys() | sort() }

  \BLOCK{ set info = genomicJSON.simple_somatic_mutations_report[gene] }
  \BLOCK{ set mutated_checkbox = docFormat.unchecked }
  \BLOCK{ set not_mutated_checkbox = docFormat.unchecked }
  \BLOCK{ set mutation_not_determined_checkbox = docFormat.unchecked }

  \BLOCK{ if info.status == "Mutated" }
  \BLOCK{ set mutated_checkbox = docFormat.checked }
  \BLOCK{ elif info.status == "Not mutated" }
  \BLOCK{ set not_mutated_checkbox = docFormat.checked }
  \BLOCK{ elif info.status == "Not determined" }
  \BLOCK{ set mutation_not_determined_checkbox = docFormat.checked }
  \BLOCK{ endif }

  \BLOCK{ if gene == "BRAF" }
  \BLOCK{ set superscript_value = "3" }
  \BLOCK{ elif gene == "NRAS" }
  \BLOCK{ set superscript_value = "4" }
  \BLOCK{ elif gene == "KRAS" }
  \BLOCK{ set superscript_value = "4" }
  \BLOCK{ endif }

  %# PROBLEM: The comment string does not work here. This is due to scoping in jinja2.
  %# Perhaps I am trying to do something too complicated, which should not be attempted here?
  \BLOCK{ set comment_string = info.alterations | map(attribute="hgvsp") | join(" ") }
%#  \BLOCK{ set comment_string = "" }
%#  \BLOCK{ for alteration in info.alterations }
%#  \VAR{comment_string}
%#  \BLOCK{ set comment_string = comment_string + " " + alteration["HGVSp"] }
%#  COMMENTSTRING:
%#  \VAR{comment_string}
%#  ENDCOMMENTSTRING.
%#  \BLOCK{ endfor }
%#  COMMENTSTRING:
%#  \VAR{comment_string}
%#  ENDCOMMENTSTRING.
%#
%#              comments = " ".join(map(lambda alteration_dict: alteration_dict["HGVSp"],
%#                                    mutn_info["Alterations"]))

  \textit{\VAR{gene}}\textsuperscript{\VAR{superscript_value}} &
  \includegraphics{\VAR{mutated_checkbox}} &
  \includegraphics{\VAR{not_mutated_checkbox}} &
  \includegraphics{\VAR{mutation_not_determined_checkbox}} & \VAR{comment_string}

  \BLOCK{ if not loop.last } \\
  \BLOCK{ endif }

  \BLOCK{ endfor }

\end{tabular}}
}}
    \end{minipage}%

\vspace{0.3cm}

\BLOCK{ if genomicJSON.purity_report.purity == "FAIL" }
  \textbf{OBSERVERA:} {\small Reducerad möjlighet att detektera tumörens mutationer samt MSI-H p.g.a. låg tumörcellshalt i aktuellt vävnadsprov.}

  \vspace{0.3cm}
\BLOCK{ endif }

\textbf{Tolkning} \par
{\small
Tilläggsinformationen från ClinSeq-panelen är av potentiellt klinisk betydelse. Analyserna är utförda på forskningsbasis med en för forskning validerad men inte kliniskt ackrediterad metod. För tolkningsstöd se nedan. \\

\textbf{Mikrosatellitinstabilitet (MSI)}\\
\textbf{\textsuperscript{1} MSI-H (MSI-high):} Tumören uppvisar höggradig mikrosatellitinstabilitet, vilket innebär inaktivering av DNA mismatch-reparation. MSI-H kan orsakas av förvärvade (somatiska) eller medfödda (ärftliga) genetiska förändringar. \\
\textbf{\textsuperscript{2} MSS/MSI-L (MSI-low):} Tumören uppvisar mikrosatellitstabilitet eller visar MSI för enstaka markörer. \\

\textbf{Kliniska situationer där MSI kan ha betydelse}\\
Tumörer med MSI-H i stadium II har en god prognos med lägre recidivrisk än tumörer med MSS/MSI-L. Fyndet bör vägas in i beslut om adjuvant cytostatikabehandling enligt nationellt vårdprogram. Vid fynd av MSI-H utan samtidig \textit{BRAF}-mutation (se nedan) bör familjeanamnes göras och ställningstagande till vidare onkogenetisk utredning tas. Syftet med en onkogenetisk utredning är att identifiera familjer med ärftlig kolorektalcancer. I dessa familjer skall särskilda kontrollprogram erbjudas för familjemedlemmar med ökad risk. \\

\textbf{Aktiverande mutationer i \textit{BRAF}, \textit{KRAS} och \textit{NRAS}}\\
\textbf{\textsuperscript{3} För \textit{BRAF}} rapporteras den mutation som med nuvarande kunskap betraktas som aktiverande (V600E). Tumörer med denna \textit{BRAF}-mutation är resistenta mot EGFR-hämmande behandling med antikropparna cetuximab och panitumumab, och denna behandling skall enligt nationellt vårdprogram då inte användas. \textit{BRAF}-mutation är vid avancerade tumörer (stadium IV) en negativ prognostisk faktor. \\
\textbf{\textsuperscript{4} För \textit{KRAS} och \textit{NRAS}} rapporteras de mutationer som med nuvarande kunskap betraktas som aktiverande (kodon 12, 13, 59, 61, 117 och 146). Tumörer med dessa \textit{KRAS}/\textit{NRAS}-mutationer är resistenta mot EGFR-hämmande behandling med antikropparna cetuximab och panitumumab, och denna behandling skall enligt nationellt vårdprogram då inte användas. \\
}
\par

\BLOCK{ endfor }

\end{document}

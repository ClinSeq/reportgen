\documentclass[10pt]{article}
\usepackage[table]{xcolor}
\usepackage{booktabs}
\usepackage[top=0.75cm,lmargin=1cm,rmargin=1cm,bottom=1cm]{geometry}
\usepackage[utf8]{inputenc}
\usepackage{graphicx}
\usepackage{longtable}
\usepackage{changepage}
\usepackage{fancyhdr}
\usepackage[many]{tcolorbox}
\usepackage{colortbl}
\usepackage{multicol}
\usepackage{multirow}
\usepackage{setspace}

\setlength{\parindent}{0pt}


\definecolor{grey}{rgb}{0.2, 0.2, 0.2}

\definecolor{lightgrey}{rgb}{0.7, 0.7, 0.7}

\definecolor{ratherlightgrey}{rgb}{0.9, 0.9, 0.9}

\renewcommand*{\familydefault}{\sfdefault}
\renewcommand{\sfdefault}{phv}

\begin{document}

%# Nesting the entire document body in a for loop, in order to generate
%# a separate report for each return address:
\BLOCK{ for address in metaJSON.return_addresses }

\pagenumbering{gobble}

\vspace{0cm}

\begin{center}
\huge{ClinSeq Analysrapport}
\end{center}

\vspace{-0.3cm}

%# Reset row colors from previous page
\rowcolors{1}{}{}
\begin{tabular}[t]{  c  c  }
  \BLOCK{ for logo in docFormat.logo_files }
    \includegraphics[width=35mm]{\VAR{logo}}
    \BLOCK{ if not loop.last } &
    \BLOCK{ endif }
  \BLOCK{ endfor }
\end{tabular}

\vspace{0cm}

\rowcolors{1}{}{}
\begin{tabular}{ l l }
\multirow{2}{10.5cm}{\begin{tabular}{l}\\[-0.12cm]
Personnummer \VAR{metaJSON.personnummer} \\
Analys genomförd \VAR{reportDate}\\
\end{tabular}} &
\multirow{4}{7cm}{
\begin{tabular}{l}\VAR{address.attn}\\
\VAR{address.line1}\\
\VAR{address.line2}\\
\VAR{address.line3}\\
\end{tabular}} \\
 & \\
 & \\
 & \\
\end{tabular}

\vspace{1.3cm}

\onehalfspacing
{
Blodprov taget \VAR{metaJSON.blood_sample_date}, remiss-ID \VAR{metaJSON.blood_referral_ID}, etikett \VAR{metaJSON.blood_sample_ID} \\
Tumörprov taget \VAR{metaJSON.tumor_sample_date}, remiss-ID \VAR{metaJSON.tumor_referral_ID}, etikett \VAR{metaJSON.tumor_sample_ID} \\
}
\par
\singlespacing

\setlength{\fboxsep}{10pt}
\setlength{\fboxrule}{3pt}

\rowcolors{1}{ratherlightgrey}{}
\fcolorbox{blue}{white}{%
  \parbox{18.9cm}{%
\centering
{\Large Randomisering till ALASCCA-studien} \par
\BLOCK{ set class_a_box = docFormat.unchecked }
\BLOCK{ set class_b_box = docFormat.unchecked }
\BLOCK{ set alascca_no_mutations_box = docFormat.unchecked }
\BLOCK{ set alascca_not_determined_box = docFormat.unchecked }
\BLOCK{ if genomicJSON.alascca_class_report.alascca_class == "Mutation class A" }
  \BLOCK{ set class_a_box = docFormat.checked }
\BLOCK{ elif genomicJSON.alascca_class_report.alascca_class == "Mutation class B" }
  \BLOCK{ set class_b_box = docFormat.checked }
\BLOCK{ elif genomicJSON.alascca_class_report.alascca_class == "Not mutated" }
  \BLOCK{ set alascca_no_mutations_box = docFormat.checked }
\BLOCK{ elif genomicJSON.alascca_class_report.alascca_class == "Not determined" }
  \BLOCK{ set alascca_not_determined_box = docFormat.checked }
\BLOCK{ endif }

\vspace{0.2cm}
\tcbox[left=0mm,right=-1mm,top=0mm,bottom=0mm,boxsep=0mm,
       boxrule=0.4pt, colframe=grey, colback=white]
{
\begin{tabular}{l l}
\includegraphics{\VAR{class_a_box}} & Mutationsklass A, patienten kan randomiseras \\
\includegraphics{\VAR{class_b_box}} & Mutationsklass B, patienten kan randomiseras \\
& \\
\includegraphics{\VAR{alascca_no_mutations_box}} & Inga mutationer, patienten kan \emph{ej} randomiseras \\
\includegraphics{\VAR{alascca_not_determined_box}} & Ej utförd/ej bedömbar, patienten kan \emph{ej} randomiseras \\
\end{tabular}
}
}
}

\vspace{0.3cm}

%# Since only alascca class is reported, only show the low purity warning if the purity is low and the alascca class is set to "Not determined".
%# I.e. no mutation was found. If mutation found and class can be reported, don't show the warning.

\BLOCK{ if (genomicJSON.purity_report.purity == "FAIL" and genomicJSON.alascca_class_report.alascca_class == "Not determined")}
  \textbf{OBSERVERA:} {\small Reducerad möjlighet att detektera tumörens mutationer p.g.a. låg tumörcellshalt i aktuellt vävnadsprov.}

  \vspace{0.3cm}
\BLOCK{ endif }

\newpage

\BLOCK{ endfor }

\end{document}

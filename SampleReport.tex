\documentclass[10pt]{article}
\usepackage[table]{xcolor}
\usepackage{booktabs}
\usepackage[top=0.75cm,lmargin=1cm,rmargin=1cm,bottom=1cm]{geometry}
\usepackage[utf8]{inputenc}
\usepackage{graphicx}
\usepackage{longtable}
\usepackage{changepage}
\usepackage{fancyhdr}
\usepackage[many]{tcolorbox}
\usepackage{colortbl}
\usepackage{multicol}
\usepackage{multirow}
\usepackage{setspace}

\setlength{\parindent}{0pt}


\definecolor{grey}{rgb}{0.2, 0.2, 0.2}

\definecolor{lightgrey}{rgb}{0.7, 0.7, 0.7}

\definecolor{ratherlightgrey}{rgb}{0.9, 0.9, 0.9}

\renewcommand*{\familydefault}{\sfdefault}
\renewcommand{\sfdefault}{phv}

\begin{document}

\pagenumbering{gobble}

\vspace{0cm}

\begin{center}
\huge{ClinSeq Analysrapport}
\end{center}

\vspace{-0.3cm}

\begin{tabular}[t]{  c  c  }
  \includegraphics[width=35mm]{./reportgen/ki-logo_cmyk_5.png} & \includegraphics[width=35mm]{./reportgen/ALASCCA_logo.png} \tabularnewline
\end{tabular}

\vspace{0.3cm}

\begin{tabular}{ l l }
\multirow{2}{10.5cm}{\begin{tabular}{l}Personnummer 121212-1212 \\Analys genomförd 2016-01-02\\\end{tabular}} & \multirow{4}{7cm}{\begin{tabular}{l}Dr Namn Namnsson\\Onkologimottagningen\\Stora Lasaretet\\123 45 Stadsby\\\end{tabular}} \\
 & \\
 & \\
 & \\
\end{tabular}

\vspace{1cm}

\onehalfspacing
{
Blodprov taget 2016-01-01, remiss-ID 98765432, etikett 12345678 \\
Tumörprov taget 2016-01-02, remiss-ID 14253647, etikett 34567890 \\
}
\par
\singlespacing

%\begin{itemize}
%   \setlength\itemsep{-1.2em}
%   \item Blodprov remiss-ID 98765432 barcode 12345678 taget 2016-01-01\\
%   \item Tumörprov remiss-ID 14253647 barcode 34567890 taget 2016-01-02\\
%\end{itemize}

\setlength{\fboxsep}{10pt}
\setlength{\fboxrule}{3pt}

\rowcolors{1}{ratherlightgrey}{}
\fcolorbox{blue}{white}{%
  \parbox{18.9cm}{%
\centering
{\Large Randomisering till ALASCCA-studien} \par
\vspace{0.2cm}
\tcbox[left=0mm,right=0mm,top=0mm,bottom=0mm,boxsep=0mm,
       boxrule=0.4pt, colframe=grey, colback=white]% set to your wish
{
  \begin{tabular}{l l}
  \includegraphics{./reportgen/rsz_unchecked_checkbox.png} & Mutationsklass A, patienten kan randomiseras \\
  \includegraphics{./reportgen/rsz_unchecked_checkbox.png} & Mutationsklass B, patienten kan randomiseras \\
  & \\
  \includegraphics{./reportgen/rsz_checked_checkbox.png} & Inga mutationer, patienten kan \emph{ej} randomiseras \\
  \includegraphics{./reportgen/rsz_unchecked_checkbox.png} & Ej utförd/ej bedömbar, patienten kan \emph{ej} randomiseras \\
  \end{tabular}
}
  }%
}

\begin{center}
\subsection*{Övrig information från ClinSeq-profil}\label{clinseq-report}
\end{center}

\vspace{-0.5cm}

\begin{minipage}{.4\linewidth}
\fcolorbox{lightgrey}{white}{%
  \parbox[t][3cm][t]{6.1cm}{%
\centering

    {\large Mikrosatellitinstabilitet\\(MSI)} \par
\vspace{0.4cm}

\tcbox[left=0mm,right=0mm,top=0mm,bottom=0mm,boxsep=0mm,
  boxrule=0.4pt, colframe=grey, colback=white]% set to your wish
{
  \begin{tabular}{l | l}
MSI-H\textsuperscript{1} & \includegraphics{./reportgen/rsz_unchecked_checkbox.png} \\
MSS/MSI-L\textsuperscript{2} & \includegraphics{./reportgen/rsz_checked_checkbox.png} \\
Ej bedömbar & \includegraphics{./reportgen/rsz_unchecked_checkbox.png} \\
  \end{tabular}
}
}}
\end{minipage}
\begin{minipage}{.6\linewidth}
\fcolorbox{lightgrey}{white}{%
  \parbox[t][3cm][t]{11cm}{%
    \centering

    \rowcolors{2}{}{ratherlightgrey}

    {\large Övriga mutationer} \par
\vspace{0.3cm}

\tcbox[left=0mm,right=0mm,top=0mm,bottom=0mm,boxsep=0mm,
  boxrule=0.4pt, colframe=grey, colback=white]% set to your wish
{
  \begin{tabular}{l | l | l | l | l}
Gen & Mutation & Ej mutation & Ej bedömbar & Kommentar \\
\arrayrulecolor{grey}\hline
\arrayrulecolor{grey}\hline
\textit{BRAF}\textsuperscript{3} & \includegraphics{./reportgen/rsz_unchecked_checkbox.png} & \includegraphics{./reportgen/rsz_checked_checkbox.png} & \includegraphics{./reportgen/rsz_unchecked_checkbox.png} &  \\
\textit{NRAS}\textsuperscript{4} & \includegraphics{./reportgen/rsz_unchecked_checkbox.png} & \includegraphics{./reportgen/rsz_checked_checkbox.png} & \includegraphics{./reportgen/rsz_unchecked_checkbox.png} &  \\
\textit{KRAS}\textsuperscript{4} & \includegraphics{./reportgen/rsz_unchecked_checkbox.png} & \includegraphics{./reportgen/rsz_checked_checkbox.png} & \includegraphics{./reportgen/rsz_unchecked_checkbox.png} &  \\
  \end{tabular}
}
}}
    \end{minipage}%

\vspace{0.3cm}

\textbf{Tolkning} \par
{\small
Tilläggsinformationen från ClinSeq-panelen är av potentiellt klinisk betydelse. Analyserna är utförda på forskningsbasis med en för forskning validerad men inte kliniskt ackrediterad metod. För tolkningsstöd se nedan. \\

\textbf{Mikrosatellitinstabilitet (MSI)}\\
\textbf{\textsuperscript{1} MSI-H (MSI-high):} Tumören uppvisar höggradig mikrosatellitinstabilitet, vilket innebär inaktivering av DNA mismatch-reparation. MSI-H kan orsakas av förvärvade (somatiska) eller medfödda (ärftliga) genetiska förändringar. \\
\textbf{\textsuperscript{2} MSS/MSI-L (MSI-low):} Tumören uppvisar mikrosatellitstabilitet eller visar MSI för enstaka markörer. \\

\textbf{Kliniska situationer där MSI kan ha betydelse}\\
Tumörer med MSI-H i stadium II har en god prognos med lägre recidivrisk än tumörer med MSS/MSI-L. Fyndet bör vägas in i beslut om adjuvant cytostatikabehandling enligt nationellt vårdprogram. Vid fynd av MSI-H utan samtidig \textit{BRAF}-mutation (se nedan) bör familjeanamnes göras och ställningstagande till vidare onkogenetisk utredning tas. Syftet med en onkogenetisk utredning är att identifiera familjer med ärftlig kolorektalcancer. I dessa familjer skall särskilda kontrollprogram erbjudas för familjemedlemmar med ökad risk. \\

\textbf{Aktiverande mutationer i \textit{BRAF}, \textit{KRAS} och \textit{NRAS}}\\
\textbf{\textsuperscript{3} För \textit{BRAF}} rapporteras den mutation som med nuvarande kunskap betraktas som aktiverande (V600E). Tumörer med denna \textit{BRAF}-mutation är resistenta mot EGFR-hämmande behandling med antikropparna cetuximab och panitumumab, och denna behandling skall enligt nationellt vårdprogram då inte användas. \textit{BRAF}-mutation är vid avancerade tumörer (stadium IV) en negativ prognostisk faktor. \\
\textbf{\textsuperscript{4} För \textit{KRAS} och \textit{NRAS}} rapporteras de mutationer som med nuvarande kunskap betraktas som aktiverande (kodon 12, 13, 59, 61, 117 och 146). Tumörer med dessa \textit{KRAS}/\textit{NRAS}-mutationer är resistenta mot EGFR-hämmande behandling med antikropparna cetuximab och panitumumab, och denna behandling skall enligt nationellt vårdprogram då inte användas. \\

\textbf{Kontakt}\\
Frågor om analysmetoden i ALASCCA-studien kan skickas till ALASCCA@meb.ki.se
}
\par

\end{document}

% TO INCLUDE ABOVE IN THE CORRECT LOCATION: / Ej Bedömbar
